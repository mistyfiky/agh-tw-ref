\documentclass[a4paper]{report}

\usepackage[utf8]{inputenc}
\usepackage[T1]{fontenc}
\usepackage{lmodern}
\usepackage[polish]{babel}
\usepackage{polski}
\usepackage{indentfirst}
\frenchspacing

\usepackage{amsmath}
\usepackage{graphicx}

\title{Komunikacja w CSP}
\author{
Maciej Szklarczyk
\and
Jan Malisz
\and
Łukasz Nośko
}
\date{2019-05-18}

\begin{document}

  \maketitle
  \thispagestyle{empty}
  \newpage

  \tableofcontents
  \newpage

  \chapter{Wstęp}

  \section{Początki CSP}
  CSP czyli Communicating Sequential Processes po raz pierwszy zostało opisane w 1978 przez Sir Charles Antony Richard Hoare’a.
  Należy pamiętać, że CSP nie jest językiem programowania - jest notacją.
  Nie posiada swojego kompilatora i środowiska do uruchamiania programów napisanych w CSP.
  Jednak należy wspomnieć, że CSP jako notacja stało się pierwowzorem dla pełnoprawnych języków programowania takich jak na przykład Occama czy GoLang.

  \section{CSP jako notacja synchroniczna}
  CSP jest notacją synchroniczną co oznacza, że komunikacja zachodzi zawsze pomiędzy dwoma procesami z których rozróżniamy dwa typy.
  Nadawcę i odbiorcę.
  Nadawca musi znać identyfikator odbiorcy i jawnie określić do kogo jest skierowana wiadomość.
  Odbiorca również musi być świadomy od kogo chce odebrać dane, aby mogło dojść do wymiany.
  Taką wymianę informacji nazywamy synchroniczną, ponieważ oba procesy o sobie wiedzą i znają swoje identyfikatory.
  Procesy jednak muszę oczekiwać na siebie, co prowadzi do tego, że odbiorca informacji będzie wstrzymany w stanie oczekującym dopóki nie otrzyma wiadomości.
  Analogicznie nadawca będzie oczekiwał na odbiorcę.
  Do komunikacji dojdzie kiedy obie strony przyjmą stan pozwalający na komunikację.

  \section{Niedeterminizm}
  Każdy projekt napisany w CSP jest zbiorem konkretnych deklaracji i instrukcji.
  Każda z tych instrukcji może wykonać się poprawnie bądź spowodować błąd.
  Jednak poprzez zdefiniowanie CSP jako notacji, a nie jako języka programowania wystąpienie błędu nie jest precyzyjnie zdefiniowane.
  Dodatkowo przez występujący niedeterminizm, niektóre instrukcje mogą powodować różne wyniki.
  Zatem dwa wykonania tego samego programu mogą zwrócić różne wyniki.

  \chapter{Składnia i semantyka}

  \section{Instrukcje}
  Instrukcje najogólniejsze pojęcie w notacji CSP.
  Dzielą się one na proste (instrukacja pusta, przypisanie, instrukcja wejścia i wyjścia) i strukrutalne.
  Tak jak było to już wspomniane, każda instrukcja w CSP wykonuje się poprawnie lub kończy się błędem.

  \section{Instrukcja pusta}
  Jest to, jak łatwo można się domyślić instrukcja która nie robi nic.
  Zawsze jej rezultatem jest powodzenie.
  Przydatna jest w instrukcjach alternatywy i iteracji.
  \begin{equation}
    \text{skip}
  \end{equation}

  \section{Przypisanie}

  \section{Założenie sekwencyjne}

  \section{Alternatywa}
  Alternatywa w CSP jest instrukcją strukturalną.
  W CSP jest niedeterministyczna.
  Składa się z oddzielonych znakiem pionowej kreski '|' instrukcji.
  Każda instrukcja, nazywana dozorowaną w najprostszej postaci składa się właśnie z dozoru, który jest wyrażeniem logicznym.
  Alternatywę rozpoczynamy od otwarcia kwadratowego nawiasu '[' a kończymy zamknięciem ']'.
  \begin{equation}
    \begin{split}
      &\lbrack x > 10; x < 20 \rightarrow y := 1\\
      &|x \leq 10 \rightarrow y := 2\\
      &|x \geq 20 \rightarrow y := 3 \rbrack
    \end{split}
  \end{equation}
  Wykonanie pojedynczej alternatywy polega na wyliczeniu wartości logicznej dozorów i w przypadku ich poprawności wykonanie instrukcji.
  W przypadku kiedy kilka dozorów jest poprawnych zostaje wykonana losowa instrukcja.
  Jeśli w przypadku poprawności jednego dozoru nie chcemy wykonać nic, używamy instrukcji 'skip'.
  Należy zadbać, żeby zawsze przynajmniej jeden dozor był prawdziwy.

  \section{Dozory złożone}
  CSP pozwala na umieszczenie w jednej instrukcji wielu dozorów.\\
  TODO\\
  Powyższy dozór nazywany jest dozorem słożonym, ponieważ jego wykonanie polega na kolejnym wyliczaniu poszczególnych warunków logicznych.
  Jeśli dozór zwróci false, cała instrukcja jest zaniechana.
  W przypadku powodzenia, następuje przejście do następnego wyrażenia logicznego.
  Jeśli wszystkie wyrażenia są poprawne to cały dozór zostaje uznany za poprawny.

  \section{Tablice dozorów}

  \section{Instrukcja pusta}

  \section{Iteracja}

  \section{Złożenia równoległe}

  \section{Tablice procesów}

\end{document}
